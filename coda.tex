%=========================================================
% codaTeX
% Modelo LaTeX para estudos de código
% v0.3 09 Abril 2022
%=========================================================

%---------------------------------------------------------
% Configura fonte, tamanho da página e tipo do documento
%---------------------------------------------------------
\documentclass[12pt,a4paper]{article}

%=========================================================
% Pacotes
%=========================================================
\usepackage[brazil]{babel}
\usepackage{listingsutf8}
\usepackage{xcolor}
\usepackage{hyperref}
\usepackage{fancyhdr}
\usepackage{graphicx}
\usepackage{graphics}
\usepackage{epstopdf}
\usepackage[output=svg]{plantuml}

%=========================================================
% Dados para a capa
%=========================================================

\title{Seu Título Aqui}
\author{Ada Lovelace}

% use \date{\today} para gerar a data atual
\date{Setembro 1843}

%=========================================================
% Configurações de imagens
%=========================================================

% Define a pasta para incorporar imagens
\graphicspath{ {./img/} }

% Define como converter os diagramas PlantUML
\epstopdfDeclareGraphicsRule{.svg}{pdf}{.pdf}{
  inkscape -z --file=#1 --export-pdf=\OutputFile
}

%=========================================================
% Estilo do documento
%=========================================================

%---------------------------------------------------------
% Destaque de sintaxe
%---------------------------------------------------------

\definecolor{verde}{rgb}{0,0.6,0}
\definecolor{cinza}{rgb}{0.5,0.5,0.5}
\definecolor{roxo}{rgb}{0.58,0,0.82}
\definecolor{fundo}{rgb}{0.95,0.95,0.92}

\lstdefinestyle{destaqueSintaxe}{
    backgroundcolor=\color{fundo},
    commentstyle=\color{cinza},
    keywordstyle=\color{roxo},
    numberstyle=\tiny\color{cinza},
    stringstyle=\color{verde},
    basicstyle=\ttfamily\footnotesize,
    breakatwhitespace=false,
    breaklines=true,
    captionpos=b,
    keepspaces=true,
    numbers=left,
    numbersep=5pt,
    showspaces=false,
    showstringspaces=false,
    showtabs=false,
    tabsize=2
}

\lstset{style=destaqueSintaxe}

%---------------------------------------------------------
% Estilo dos links
%---------------------------------------------------------

\hypersetup{
    colorlinks=true,
    linkcolor=teal,
    filecolor=teal,
    urlcolor=teal
    }

%---------------------------------------------------------
% Estilo da capa
%---------------------------------------------------------
\fancypagestyle{plain}{%
    \fancyhf{}%
    \fancyhead{}%
}

%---------------------------------------------------------
% Estilo do corpo
%---------------------------------------------------------
\pagestyle{fancy}
\fancyhf{}
\fancyhead[R]{\thepage}
\renewcommand{\headrulewidth}{0pt}

%=========================================================
% INÍCIO DO DOCUMENTO
%---------------------------------------------------------
\begin{document}
%=========================================================

\maketitle % Gera o título

%---------------------------------------------------------
\begin{quote}
Texto de exemplo. Nunc imperdiet mauris dolor, vel ultricies leo sagittis quis. Nam volutpat lectus ac turpis ultricies, ac fermentum risus scelerisque. Donec vel molestie ex. Integer augue sapien, mollis nec aliquet id, congue at ipsum.
\end{quote}

%---------------------------------------------------------
\begin{minipage}{\textwidth}
%=========================================================
\section*{Situação 1}
%=========================================================

Faça um programa que receba dois números, calcule e mostre a subtração do primeiro número pelo segundo.

%---------------------------------------------------------
\subsection*{Variáveis}
%---------------------------------------------------------
\begin{itemize}
    \item entrada: \verb|n1|, \verb|n2|
    \item processamento: \verb|subtracao <- n1 - n2|
    \item saída: \verb|subtracao|
\end{itemize}

%---------------------------------------------------------
\subsection*{Fluxograma}
%---------------------------------------------------------

% Incorporar imagem da pasta img
%\includegraphics[scale=1]{imagem}

% Gerar diagrama com PlantUML
\begin{plantuml}
    @startuml
    !theme plain
    start
    :n1, n2: int/
    :subtracao <- n1 - n2]
    :"Resultado: ", subtracao/
    stop
    @enduml
\end{plantuml}

%---------------------------------------------------------
\subsection*{Pseudocódigo}
%---------------------------------------------------------
\begin{lstlisting}[language=C++]

    // exemplo de comentário

    var n1, n2, subtracao: inteiro
    escreva("Insira o primeiro número: ")
    leia(n1)
    escreva("Insira o segundo número: ")
    leia(n2)
    subtracao <- n1 - n2
    escreva("Resultado: ", subtracao)
    
    // testes de caracteres especiais:
    // ã â á à ê é í ó ô ú ç Ã Â Á À Ê É Í Ó Ô Ú Ç
    
\end{lstlisting}

%---------------------------------------------------------
\end{minipage}

%=========================================================
\end{document}
